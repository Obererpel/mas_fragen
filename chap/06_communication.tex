\section{Communication}
\subsection{Why need agents to communicate? What are the problems in open communities?}
\begin{itemize}
	\item Austausch von Wissen
	\item Koordination von Agenten, Verteilung von Aufgaben
	\item Kommunikation von Teillösungen
	\item Open Communities: ???TODO???
\end{itemize}
\subsection{KQML and KIF what is their purpose,}
\begin{itemize}
	\item KQML: Knowledge Query and Manipulation Language
	\item KIF: Knowledge Interchange Format
	\item KQML ist eine äußers Protokoll für den Austausch von Wissen
	\item KIF: eine Sprache für KQML
	\item KIF erlaubt Ausdrücke für PL1 in Form von S-Expressions + spez. Ausdrücke für Mengen usw.
\end{itemize}
\subsection{How does ACL looks like (and how are both KGML and KIF reflected in those)?}
\begin{itemize}
	\item ACL = Agent Communication Language (Sprache für die Kommunikation von Agentens)
	\item BSP: KQML, FIPAs
	\item Bestandteile: Performative (ask-if, perform, tell, reply...)+ Inhalt
		\begin{itemize}
			\item Absender
			\item Empfänger
			\item Inhalt
			\item Ontologie
			\item Angabe einer Sprache für den Nachrichteninhalt (LISP,Prolog und co)
		\end{itemize}
	\item Standardisierungsforschlag von Foundation for Intelligent Physical Agents (FIPA)
\end{itemize}
\subsection{What is an ontology and why do we need them?}
gemeinsamer Kontext wie Begriffe in der Nachricht zu interpretieren sind (Addresse im Sinne von Web vs. im Sinne von Personen)
\subsection{What does illocutionary aspect of communication refer to and how has this been reflected in ACL?}
%TODO What does illocutionary aspect of communication refer to and how has this been reflected in ACL?