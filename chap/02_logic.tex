\section{Logic}
\subsection{How can we use logic to infer an action?}
\begin{itemize}
	\item Logische Agenten: Wissensbasis + Regelmenge zum Schließen (Aussagenlogik, Prädikatenlogik, Modallogik usw)
	\item Beobachtung der Umwelt und Generierung der Perception
	\item Aktualisierung der Wissensbasis
	\item Überprüfe ob eine Aktion direkt abgeleitet werden kann $\delta \vDash do(a)$, führe a aus falls a abgeleitet werden kann
	\item Falls keine Aktion gefunden: Überprüfe ob eine Aktion nicht direkt ausgeschlossen werden kann: $\delta \not \vDash \neg do(a)$, für a aus falls a nicht direkt ausgeschlossen werden kann
\end{itemize}
\subsection{How does this relate to BDI agents?}
\begin{itemize}
	\item BDI-Agenten: Beliefes (Wissensbasis über Welt, eigene Fähigkeiten und anderen Agenten), Disires (Ziele welche längerfristig erreicht werden sollen), Intention (Teilziel welches aktuell verfolgt wird)
	\item Definition der Semantik von B,D,I und deren Zusammenhänge auf Basis von Modallogik + CTL
\end{itemize}
\subsection{What is modal logic, the kripke semantic?}
\begin{itemize}
	\item Kripke-Struktur: $<W,R,\mu>$
	\item W eine Menge von Welten
	\item R eine Übergangsrelation zwischen den Welten
	\item $\mu$ eine Funktion welcher jeder logischen Variable ein Wert zu weist
	\item Modallogik: neue Operatoren $\square$(notwendigerweise), $\diamond$(möglicherweise)
	\item Definition der Operatoren auf den Welten
	\item $<M,w>\vDash\square p$: falls für jedes $(w,w') \in R$ gilt $<M,w'> \vDash p$ (Wenn p ausgehend von der aktuellen Welt w  in jeder erreichbaren Welt $w'$ wahr ist)
	\item $<M,w>\vDash\diamond p$: falls für ein $(w,w') \in R$ gilt $<M,w'> \vDash p$ (Wenn p ausgehend von der aktuellen Welt w mindestens eine Welt erreichbar ist wo $w'$ gilt)
	\item Semantik von modal Logik hängt von den geforderten Axiomen ab
	\end{itemize}
%TODO What is modal logic, the kripke semantic?
\subsection{What does positiv introspection or negative introspection mean?}
\begin{itemize}
	\item gilt für Wissen von Agenten über mehrere Agenten
	\item $K_i p$: Agent i weiss p
	\item positv: $K_i p \rightarrow K_i K_i p$ (Wenn Agent p glaubt, weiss dieser das er p glaubt)
	\item negativ: $\not K_i \not p \rightarrow K_i \not K_i \not p$ (Wenn Agent nicht p glaubt, weiss dieser das er  nicht p glaubt; ein Agent weiss was er nicht weiss)
\end{itemize}
%TODO What does positiv introspection or negative introspection mean?
\subsection{Given the following example, what can an agent deduce about a situation based on modal logic?}
%TODO Given the following example, what can an agent deduce about a situation based on modal logic?
\subsection{Explain the relation between rational agents and logic specifications? So what to expect!}
%TODO Explain the relation between rational agents and logic specifications? So what to expect!