\section{Introduction}
\subsection{Limitations of the traditional AI approach that motivated research on agents?}
\begin{itemize}
	\item Traditionelle AI: Symbolisches Schließen (keine Verteilung der Aufgaben)
	\item Agenten erlauben höhere Flexibilität und agieren autonom
	\item Agenten erlauben Verteilung der Aufgaben
\end{itemize}
\subsection{What is an agent or is it simply a programm?}
%CHECK What is an agent or is it simply a programm?
\label{subsec:agentvsprogram}
\begin{itemize}
	\item Ein Agent ist ein Computersystem/eine Software, welche(s) unabhängig (autonom) agieren kann, um Nutzerwünsche zu erfüllen
	\item Ein Agent ist \begin{description}
		\item[Situiert] Eingebettet in einer Umgebung
		\item[Autonom] Fähigkeit zu selbstständigem Handeln
		\item[Reaktiv] Fähigkeit auf Umgebungsänderungen zu reagieren
		\item[Proaktiv] Fähigkeit zu zielgerichtetem Verhalten
		\item[Intelligent] Fähigkeit zu rationalem Handeln
		\item[Sozial] Fähigkeit zur Komminikation
	\end{description}
\end{itemize}
\subsection{What are the central areas of expertises and science that contributed to MAS, give examples}
%CHECK What are the central areas of expertises and science that contributed to MAS,
%TODO give examples
Multiagentensysteme sind eine Schnittmenge der Felder \textbf{Spieltheorie} (Finden von Strategien), \textbf{Künstliche Intelligenz}(Planning), \textbf{Soziologie}(Kommunikation zwischen Agenten, Gruppenstruktur von Agenten) und \textbf{Verteilte Systeme} (Mobile Agents).
\subsection{What does time and ressource boundedness mean?}
%TODO What does time and ressource boundedness mean?
\begin{itemize}
	\item Ein Agent hat nicht genug Resourcen um ein Problem alleine zu lösen
	\item Aufteilung des Problems in Teilprobleme
	\item Regulierung der Güte der Lösungen
\end{itemize}
\subsection{Difference between agents and objects?}
%CHECK Difference between agents and objects?
\emph{Objects do it for free, agents do it for money.}
\begin{description}
	\item [Objekte] \begin{itemize}
		\item sind passiv, sie haben keine Kontrolle über die Aktivierung von Methoden
		\item sind für ein allgemeines Ziel entworfen worden
		\item sind typischerweise in einen Thread integriert
	\end{itemize}
	\item [Agenten] \begin{itemize}
		\item handeln autonom
		\item haben verschiedene Ziele
		\item haben ihren eigenen Thread
	\end{itemize}
\end{description}
\subsection{Properties of agents?}
\begin{description}
		\item[Situiert] Eingebettet in einer Umgebung
		\item[Autonom] Fähigkeit zu selbstständigem Handeln
		\item[Reaktiv] Fähigkeit auf Umgebungsänderungen zu reagieren
		\item[Proaktiv] Fähigkeit zu zielgerichtetem Verhalten
		\item[Intelligent] Fähigkeit zu rationalem Handeln
		\item[Sozial] Fähigkeit zur Komminikation
	\end{description}
Siehe \ref{subsec:agentvsprogram}.