\section{Decision Theory}
\subsection{How does decision theory relate to MAS?}
\begin{itemize}
	\item Mehere Agenten
	\item Spieltheorie für die Untersuchung von Verhalten in Begegnungen
\end{itemize}
%TODO How does decision theory relate to MAS?
\subsection{The difference between Nash Equilibrium and Pareto Optimum?}
%TODO The difference between Nash Equilibrium and Pareto Optimum?
Zwei Strategien $s_i$ und $s_j$ befinden sich im \textbf{Nash-Gleichgewicht}, falls
\begin{enumerate}
	\item gegeben der Annahme, dass Agent $i$ Strategie $s_i$ verfolgt, Agent $j$ am Besten Strategie $s_j$ verfolgt \textbf{und}
	\item gegeben der Annahme, dass Agent $j$ Strategie $s_j$ verfolgt, Agent $i$ am Besten Strategie $s_i$ nutzt.
\end{enumerate}
Eine Zustand $s$ (resultierend aus den Strategien der Agenten) ist \textbf{Pareto-Optimal}, wenn es keinen anderen Zustand $s'$ gibt, in dem a) jeder einzelne Agent mindestens den selben Gewinn hat, wie in $s$ und b) mindestens ein Agent mehr Gewinn erzielt. 

\begin{itemize}
	\item Das Nash-Gleichgewicht bezieht sich auf ein Paar von \textbf{Strategien}, Pareto-Optimalität hingegen auf einen \textbf{konkreten Ausgang}.
	\item Es gibt nicht immer ein Nash-Gleichgewicht, pareto-optimale Ausgänge hingegen schon.
	\item Ein Zustand, der durch die Wahl zweier Strategien im Nash-Gleichgewicht entsteht, ist nicht zwangsläufig pareto-optimal und umgekehrt. 
\end{itemize}
\subsection{Given the current example, what is the Nash Equilibrium what the pareto Optimum?}
%TODO Given the current example, what is the Nash Equilibrium what the pareto Optimum?
\begin{table}[]
	\centering
	\label{my-label}
	\begin{tabular}{ll|l|l|}
		&        & \multicolumn{2}{c|}{i} \\
		&        & defect      & coop     \\ \hline
		\multirow{2}{*}{j} & defect & 1 1         & 5 0      \\ \cline{2-4} 
		& coop   & 0 5         & 3 3      \\ \hline
	\end{tabular}
	\caption{Payoff Matrix: Prisoners Dilemma}
\end{table}
Die Strategien $(d,d)$ sind im Nash-Gleichgewicht, weil $i$, wenn $j$ die Strategie $d$ spielt den höchsten Gewinn mit der Strategie $d$ (1) hat und umgekehrt.
Dieser Ausgang ist nicht pareto-optimal, da die Strategien (c,c) für beide eine höhere Auszahlung bringen.

Der Ausgang der Strategien $(d,c)$ ist pareto-optimal, da $j$ in allen anderen Ausgängen einen geringeren Payoff, als 5 hat. Analog dazu $(c,d)$.

$(c,c)$ ist pareto-optimal, da $(d,c)$ und $(c,d)$ jeweils für mindestens einen Teilnehmer einen Payoff von 0 (< 3) bedeutet und $(d,d)$ für beide weniger Gewinn bringt.
\subsection{How can we derive different strategies from Pay Off matrices?}
%TODO How can we derive different strategies from Pay Off matrices?

\subsection{Prisoners Dilemma why is it interesting?}
