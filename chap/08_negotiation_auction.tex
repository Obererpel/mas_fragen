\section{Negotiation Auction}
\subsection{Why do agents need to negotiate?}
%CHECK Why do agents need to negotiate?
\begin{itemize}
	\item Häufig gibt es Ressourcen, die ein Agent besitzt, die jedoch andere Agenten auch haben wollen.
	\item Da die Ressource nur an einen (oder wenige) Agenten weitergegeben werden kann, muss dieser eine irgendwie bestimmt werden.
	\item Weiterhin können sich Agenten Aufgaben untereinander aufteilen.
\end{itemize}
\subsection{The desirable properties of the rules of encounters, give example what are the implications of the different auctions?}
%CHECK The desirable properties of the rules of encounters, give example what are the implications of the different auctions?
Geklaut aus der Zsf.
\begin{multicols}{2}
\subsubsection{Englische Auktion}
\begin{itemize}
	\item höchstes Gebot gewinnt, für alle sichtbar, aufsteigend
	\item Dominante Strategie: minimale Erhöhung des höchsten Gebotes bis Obergrenze erreicht, falls diese überschritten: Rückzug
	\item Anfällig: Fluch des Gewinners (Bezahlt meistens zuviel), Lockvögel (Agent arbeitet mit Auktionator zusammen und treibt den Preis künstlich in die Höhe)
\end{itemize}
\subsubsection{Holländische Auktion}
\begin{itemize}
	\item offene Gebote, absteigend
	\item Auktionator startet mit hohen Startgebot
	\item Auktionator senkt Preis bis Agent ein Gebot zu dem Preis abgiebt
	\item Gewinner: Agent mit der Preisabgabe
\end{itemize}
\subsubsection{First-Price Sealed-Bid Auction}
\begin{itemize}
	\item einmaliges Gebot, verborgen
	\item eine Runde
	\item Bieter sendet Gebot
	\item Bieter mit höchstem Gebot gewinnt
	\item Gewinner bezahlt Preis des höchstem Gebotes
	\item Beste Strategie: biete weniger als der eigentlich wert
\end{itemize}
\subsubsection{Vickrey Auction}
\begin{itemize}
	\item second preis, sealed bit
	\item Gewinner mit dem höchstem Gebot zum Preis vom zweit höchstem Gebot
	\item Beste Strategie: Preisabgabe zum wirklichen Wert
	\item überbieten wird dominiert durch bieten des echten Wertes: Wenn Bieter höheren Wert als die anderen gewinnt dieser obwohl er überboten hat; wenn Bieter niedrigeren Wert als andere Bieter hat er verloren obwohl überboten hat oder nicht
	\item unterbieten wird dominiert durch bieten des echten Wertes: wenn Bieter zu gering verliert er; wenn Bieter zu hoch wird er gewinnen
	\item anfällig für antisoziales Verhalten
	\begin{itemize}
		\item Agent A beziffert den Wert bei 90\$ und weiß, dass Agent B 100\$ bieten würde.
		\item Agent A kann Zuschlag also nicht bekommen.
		\item Stattdessen bietet Agent A 99\$ um den Preis für Agent B in die Höhe zu treiben und diesem so zu schaden.
	\end{itemize}
\end{itemize}
\subsection{Probleme}
\begin{itemize}
	\item Auktionen sind anfällig für Lügen vom Auktionator und Absprache von Bietern
	\item alle Auktionen sind können manipuliert werden durch Absprache der Bieter
	\item ein böser Auktionator kann bei der Vickrey Auktion lügen beim zweit höchsten Gebot
	\item Shills (Lockvögel) können bei der Englischen Auktion den Preis in die Höhe treiben
	\item Anwendung von Auktionieren: Lastverteilung, Routing, Koordination
\end{itemize}
\end{multicols}
\subsection{What is an task-oriented domain and how can we find a good strategy?}
%TODO What is an task-oriented domain and how can we find a good strategy?
\begin{itemize}
	\item Eine TOD ist ein Tripel $<T,Ag,c>$ mit \begin{itemize}
		\item Menge $T$ aller möglichen Aufgaben
		\item Menge $Ag=\{1..n\}$ aller teilnehmenden Agenten
		\item Die Kostenfunktion $c = \mathfrak{P}(T)\rightarrow\mathbb{R}^+$ für Mengen von Aufgaben.
	\end{itemize}
	\item Mehrere Agenten haben mehrere Aufgaben, die erledigt werden müssen.
	\item 
\end{itemize}
\subsection{What do we need to consider in negotiations, e.g. referring to lying.}
%TODO What do we need to consider in negotiations, e.g. referring to lying.
\subsection{Zeuthen strategy explain?}
%TODO Zeuthen strategy explain?